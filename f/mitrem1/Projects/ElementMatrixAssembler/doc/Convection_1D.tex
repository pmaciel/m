%% LyX 1.4.2 created this file.  For more info, see http://www.lyx.org/.
%% Do not edit unless you really know what you are doing.
\documentclass[english]{article}
\usepackage[T1]{fontenc}
\usepackage[latin1]{inputenc}
\usepackage{geometry}
\geometry{verbose,letterpaper,lmargin=2cm,rmargin=2cm}

\makeatletter

%%%%%%%%%%%%%%%%%%%%%%%%%%%%%% LyX specific LaTeX commands.
%% Bold symbol macro for standard LaTeX users
\providecommand{\boldsymbol}[1]{\mbox{\boldmath $#1$}}


%%%%%%%%%%%%%%%%%%%%%%%%%%%%%% User specified LaTeX commands.
%% LyX 1.4.2 created this file.  For more info, see http://www.lyx.org/.
%% Do not edit unless you really know what you are doing.





\makeatletter

%%%%%%%%%%%%%%%%%%%%%%%%%%%%%% LyX specific LaTeX commands.
%% Bold symbol macro for standard LaTeX users



%%%%%%%%%%%%%%%%%%%%%%%%%%%%%% User specified LaTeX commands.
%% LyX 1.4.2 created this file.  For more info, see http://www.lyx.org/.
%% Do not edit unless you really know what you are doing.





\makeatletter

%%%%%%%%%%%%%%%%%%%%%%%%%%%%%% LyX specific LaTeX commands.
%% Bold symbol macro for standard LaTeX users




\makeatother



\makeatother

\usepackage{babel}
\makeatother
\begin{document}

\title{Convection 1D}

\maketitle

\section{Equations}


\subsection{Mass conservation}

\begin{equation}
\frac{\partial c_{i}}{\partial t}=-\vec{\nabla}.\vec{N}_{i}+R_{i}\label{eq:mass conservation}\end{equation}
 with\begin{equation}
\vec{N}_{i}=c_{i}\vec{v}-\sum_{j}D_{ij}\vec{\nabla}c_{j}-w_{i}c_{i}\vec{\nabla}U\label{eq:flux}\end{equation}



\subsection{Poisson's equation}

\begin{equation}
\vec{\nabla}^{2}U+\frac{F}{\epsilon}\sum_{i}z_{i}c_{i}=0\label{eq:Poisson}\end{equation}



\subsection{Butler-Volmer kinetics}

\begin{equation}
v=k_{ox}\exp\left[\frac{\alpha_{ox}nF}{RT}\left(V-U\right)\right]c_{red}-k_{red}\exp\left[-\frac{\alpha_{red}nF}{RT}\left(V-U\right)\right]c_{ox}\label{eq:Butler-Volmer}\end{equation}



\section{Element matrix}


\subsection{Convection\label{sub:Convection matrix}}


\subsubsection{Fluctuation in node}

\[
\Delta c_{i}^{m}=\sum_{e}\alpha_{e}^{m}\Phi_{e}\]



\subsubsection{Element contribution to fluctuation in node }

\[
\begin{array}{rcl}
\Phi & = & \int_{L}-\vec{v}.\vec{\nabla}c_{i}dL\\
 & = & -\left(\vec{n}^{1}c_{i}^{1}+\vec{n}^{2}c_{i}^{2}\right).\vec{v}_{av}\\
 & = & -\left(k^{1}c_{i}^{1}+k^{2}c_{i}^{2}\right)\end{array}\]


\begin{itemize}
\item One target (e.g. node 1) 
\end{itemize}
\[
\alpha^{1}=1\]


\[
\alpha^{2}=0\]


\begin{equation}
\left\{ \begin{array}{c}
\Delta c_{i}^{1}\\
\Delta c_{i}^{2}\end{array}\right\} =\left[\begin{array}{cc}
-k^{1} & -k^{2}\\
0 & 0\end{array}\right]\left\{ \begin{array}{c}
c_{i}^{1}\\
c_{i}^{2}\end{array}\right\} \label{eq:element matrix convection}\end{equation}



\subsubsection{Example: binary electrolyte}

\begin{itemize}
\item One target (node 1) 
\end{itemize}
\begin{equation}
\left\{ \begin{array}{c}
\Delta c_{A}^{1}\\
\Delta c_{B}^{1}\\
\Delta U^{1}\\
\Delta c_{A}^{2}\\
\Delta c_{B}^{2}\\
\Delta U^{2}\end{array}\right\} =\left[\begin{array}{cccccc}
-k^{1} & 0 & 0 & -k^{2} & 0 & 0\\
0 & -k^{1} & 0 & 0 & -k^{2} & 0\\
0 & 0 & 0 & 0 & 0 & 0\\
0 & 0 & 0 & 0 & 0 & 0\\
0 & 0 & 0 & 0 & 0 & 0\\
0 & 0 & 0 & 0 & 0 & 0\end{array}\right]\left\{ \begin{array}{c}
c_{A}^{1}\\
c_{B}^{1}\\
U^{1}\\
c_{A}^{2}\\
c_{B}^{2}\\
U^{2}\end{array}\right\} \label{eq:element matrix convection example}\end{equation}



\section{Element jacobian}


\subsection{Convection\label{sub:Convection jacobian}}

Zero contribution.
\end{document}
