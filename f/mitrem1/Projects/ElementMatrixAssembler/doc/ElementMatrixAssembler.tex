%% LyX 1.4.2 created this file.  For more info, see http://www.lyx.org/.
%% Do not edit unless you really know what you are doing.
\documentclass[english]{article}
\usepackage[T1]{fontenc}
\usepackage[latin1]{inputenc}

\makeatletter

%%%%%%%%%%%%%%%%%%%%%%%%%%%%%% LyX specific LaTeX commands.
%% Bold symbol macro for standard LaTeX users
\providecommand{\boldsymbol}[1]{\mbox{\boldmath $#1$}}


\usepackage{babel}
\makeatother
\begin{document}

\title{ElementMatrixAssembler library}

\maketitle

\section{Equations and nomenclature}


\subsection{Stationary mass conservation}

\[
-\vec{\nabla}.\vec{N}_{i}+R_{i}=0\]


with

\[
\vec{N}_{i}=c_{i}\vec{v}-\sum_{j}D_{ij}\vec{\nabla}c_{j}-w_{i}c_{i}\vec{\nabla}U\]


\begin{itemize}
\item $D_{ij}$ is the diffusion factor.
\item $w_{i}c_{i}$ is the migration factor.
\end{itemize}

\subsection{Electrostatics}

\[
\vec{\nabla}^{2}U+\frac{F}{\epsilon}\sum_{i}z_{i}c_{i}=0\]


\begin{itemize}
\item The electrostatics potential factor is $1$ if you use Poisson and
$0$ if you use electroneutrality.
\item $\frac{z_{i}F}{\epsilon}$ is the electrostatics concentration factor
if you use Poison and $z_{i}$ if you use electroneutrality.
\end{itemize}

\subsection{Butler-Volmer kinetics}

\[
v=k_{ox}\exp\left[\frac{\alpha_{ox}nF}{RT}\left(V-U\right)\right]c_{red}-k_{red}\exp\left[-\frac{\alpha_{red}nF}{RT}\left(V-U\right)\right]c_{ox}\]



\section{Include}

Additional Include Directories: ElementMatrixAssembler\textbackslash{}include 

In your code: \#include \char`\"{}ElementMatrixAssembler.h\char`\"{}


\section{Functions}


\subsection{Constructor}

Pass the number of dimensions, the MITReM object and the names of
the numerical schemes to use.


\subsubsection{Convection schemes}

\begin{itemize}
\item Empty
\item N
\item LDA
\end{itemize}

\subsubsection{Diffusion schemes}

\begin{itemize}
\item Empty
\item DualMesh
\end{itemize}

\subsubsection{Migration schemes}

\begin{itemize}
\item Empty
\item DualMesh
\end{itemize}

\subsubsection{Homogeneous reaction schemes}

\begin{itemize}
\item Empty
\item DualMesh
\end{itemize}

\subsubsection{Electrostatics schemes}

\begin{itemize}
\item Empty
\item DualMesh
\end{itemize}

\subsubsection{Time schemes}

\begin{itemize}
\item Empty
\item DualMesh
\end{itemize}

\subsubsection{Electrochemical reaction schemes}

\begin{itemize}
\item Empty
\item DualMesh
\end{itemize}

\subsection{calcElementMat}

Returns the element matrix. Pass the coordinates, the velocities,
the concentrations, the potentials, the temperatures, the densities
and the void fractions in the nodes of the element.


\subsection{calcElementJac}

Returns the element jacobian. Pass the coordinates, the velocities,
the concentrations, the potentials, the temperatures, the densities
and the void fractions in the nodes of the element.


\subsection{calcBoundaryElementVec}

Returns the boundary element vector. Pass the coordinates, the concentrations,
the potentials, the temperatures, the densities and the void fractions
in the nodes of the boundary element, and also the list of electrochemical
reaction indices and the length of this list and the electrode potential.


\subsection{calcBoundaryElementJac}

Returns the boundary element jacobian. Pass the coordinates, the concentrations,
the potentials, the temperatures, the densities and the void fractions
in the nodes of the boundary element, and also the list of electrochemical
reaction indices and the length of this list and the electrode potential.
\end{document}
